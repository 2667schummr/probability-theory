\documentclass[12pt]{article}
\usepackage[margin=1in]{geometry}
\usepackage{amsmath}
\usepackage{amssymb}
\usepackage[mathscr]{euscript}
\newcommand{\Sd}{\mathcal{S}_d}
\newcommand{\U}{\mathcal{U}}


\title{Theory of Probability Solutions} 

\author{Ryan Schumm, PhD}

\date{} 

\begin{document}

    \maketitle
    \setlength{\parindent}{0pt}

    \section*{Durrett ed. 4, Problem 1.1.3}
    Let $\mathcal{S}_d = \left\{(a_1, b_1]\times\cdots\times(a_d, b_d]: a_i, b_i\in\mathbb R, \ 1 \leq i \leq d\right\}$. Show that 
    $\sigma\left(\mathcal S_d\right) = \mathcal R^d$ where $\mathcal R^d$ is the Borel subsets of $\mathbb R^d$.\\

    \textbf{Solution:}\\
    Let $\U$ be the set of all open sets of $\mathbb R^d$ and $S\in\Sd$. By definition, we can write
    \begin{align*}
        S = (a_1, b_1]\times\cdots\times(a_d, b_d], \ a_i, b_i\in(-\infty, \infty).
    \end{align*}
    Take $\eta_i > 0$ such that $a_i + \eta_i < b_i$ for all $i = 1,\ldots,d$ and set 
    \begin{align}
        &U = (a_1, b_1)\times\ldots\times(a_d, b_d),\\
        &V = [a_1 + \eta_1, b_1]\times\ldots\times[a_d + \eta_d, b_d]
    \end{align}
    It follows that $S = U \cup V\in\sigma(\U)$ since the compliment of any open set is a closed set and sigma algebras are closed under unions. By definition 
    of a generated sigma algebra and since $S$ was arbitrary, we have $\Sd\subset\sigma(\U)\implies\sigma(\Sd)\subset\sigma(\U)$.\\

    Now let $U\in\U$ and define
    \begin{align}
        \label{rat-recs}
        \mathscr S = \left\{S = (a_1, b_1]\times\cdots\times(a_d, b_d]: \ a_i, b_i\in\mathbb Q, \ S\subset U \right\}.
    \end{align}
    Note that $\mathscr S$ is a countable subset of $\Sd$ which implies that $\cup_{S\in \mathscr S}S\in \sigma(\Sd)$. For any $x \in U$, we have that 
    \begin{align}
        x\in(c_1, e_1)\times\cdots\times(c_d, e_d)\subset U
    \end{align}
    since the open sets $\U$ are generated by open boxes. Since $\mathbb Q$ is dense in $\mathbb R$, there exists $a_i, b_i\in\mathbb Q$ such that 
    $x_i\in(a_i, b_i]\subset (c_i, e_i)$ for all $i=1,\ldots,d$. It follows that 
    \begin{align}
        x\in\cup_{S\in \mathscr S}S\implies U\subset\cup_{S\in \mathscr S}S
    \end{align}
    since $x$ was
    arbitrary. By construction, we also have that $\cup_{S\in \mathscr S}S\subset U \implies U = \cup_{S\in \mathscr S}S$. Therefore, we have shown that
    $\U\subset \sigma(\Sd)\implies \sigma(\U)\subset \sigma(\Sd)$. Hence, $\sigma(\Sd) = \sigma(\U) = \mathcal R^d$.
    
    \section*{Durrett ed. 4, Problem 1.1.4}
    A sigma field $\mathcal F$ is said to be countably generated if there is a countable collection of subsets $\mathscr C\subset \mathcal F$
    so that $\sigma(\mathscr C) = \mathcal F$. Show that $\mathcal R^d$ is a countably generated.\\

    \textbf{Solution:}\\
    Let the countable set $\mathscr C$ be given by
    \begin{align}
        \mathscr C = \left\{(a_1, b_1]\times\cdots\times(a_d, b_d]: \ a_i, b_i\in\mathbb Q\right\}
    \end{align}
    and observe that $\mathscr C\subset\Sd\implies\sigma(\mathscr C)\subset\sigma(\Sd) = \mathcal R^d$. In the previous problem, we showed that
    $U = \cup_{S\in\mathscr S}S\in\mathscr C$ for all $U\in\U$ where $\mathscr S\subset \mathscr C$ is defined in (\ref{rat-recs}). 
    It follows that $\U\subset\mathscr C\implies \sigma(\U) = \mathcal R^d \subset\sigma(\mathscr C)$.
    Thus, we have shown that $\mathcal R^d$ is generated by the countable set $\mathscr C$.

    \section*{Durrett ed. 4, Problem 1.2.3}
    Show that a distribution function at most countably many discontinuities.\\

    \textbf{Solution:}\\
    Let $X$ be a random variable with distribution function $F$ and $D\subset \mathbb R$ be the set of discontinuities of $F$.
    Since $\mathbb Q$ is dense in $\mathbb R$ and $F$ is an increasing function with $F(d) = F(d^+)$, there exists a $q_d\in\mathbb Q$ such that 
    $F(d^-) < q_d < F(d)$ for all $d\in D$. Furthermore, for any $x, y\in D$ with $x < y$, we have that $F(x^-) < F(x) \leq F(y^-) < F(y)$.
    Therefore, $q_x \neq q_y$. It follows that the function $f: D\to\mathbb Q$ where $f(d) = q_d, \ \forall d\in D$ is injective
    which implies $|D| \leq |\mathbb Q|$. Since $\mathbb Q$ is countable, $D$ is at most countable.

    \section*{Durrett ed. 4, Problem 1.2.5}
    Suppose $X$ has continuous density $f$, $P(\alpha \leq X \leq \beta) = 1$ and $g$ is a function that is strictly increasing and differentiable on $(\alpha, \beta)$.
    Then $g(X)$ has density $f(g^{-1}(y)) / g'(g^{-1}(y))$ for $y\in(g(\alpha), g(\beta))$.\\

    \textbf{Solution:}\\
    By the definition of a distribution function and the fact that $g$ is strictly increasing function, we can write
    \begin{align}
        P(g(X) \leq z) = P(X \leq g^{-1}(z)) = F(g^{-1}(z)) = \int_{\alpha}^{g^{-1}(z)}f(x)dx.
    \end{align}
    Set $\xi = g(x)\implies dx = d\xi/g'(g^{-1}(\xi))$ and observe that 
    \begin{align}
        \int_{\alpha}^{g^{-1}(z)}f(x)dx = \int_{g(\alpha)}^z\frac{f(g^{-1}(\xi))}{g'(g^{-1}(\xi))}d\xi
    \end{align}
    which yields the desired result.

\end{document}
